\input{preambuloSimple.tex}

%----------------------------------------------------------------------------------------
%	TÍTULO Y DATOS DEL ALUMNO
%----------------------------------------------------------------------------------------

\title{	
\normalfont \normalsize 
\textsc{\textbf{Estructura de Computadores (2016-2017)} \\ Subgrupo C3 \\ Grado en Ingeniería Informática\\ Universidad de Granada} \\ [25pt] % Your university, school and/or department name(s)
\horrule{0.5pt} \\[0.4cm] % Thin top horizontal rule
\huge Práctica 3: Programación	mixta	C-asm	x86	Linux \\ % The assignment title
\horrule{2pt} \\[0.5cm] % Thick bottom horizontal rule
}

\author{Mario Rodríguez Ruiz} % Nombre y apellidos

\date{\normalsize\today} % Incluye la fecha actual

%----------------------------------------------------------------------------------------
% DOCUMENTO
%----------------------------------------------------------------------------------------

\begin{document}

\maketitle % Muestra el Título

\newpage %inserta un salto de página

\tableofcontents % para generar el índice de contenidos

\listoffigures

\newpage

%----------------------------------------------------------------------------------------
%	Diario de trabajo
%----------------------------------------------------------------------------------------
\section{Diario de trabajo}
\subsection {Primera sesión}
Finalizar el tutorial de la práctica 3. 
Ejecutar las distintas versiones del programa \textbf{suma}:

La primera versión consistía en modificar el código para que funcionara
según \textbf{cdecl}.

La segunda añadía la llamada \textbf{printf()} para sacar por pantalla el 
resultado en decimal y hexadecimal que sustituye la llamada directa al kernel Linux.

La tercera eliminaba cualquier rastro de ensamblador en \textbf{suma()} creando una nueva equivalencia completa en \textbf{C}.

La cuarta dejaba como única instrucción un salto a la subrutina suma.

La quinta era dejar completamente el código en \textbf{lenguaje C}.

La sexta consistía en volver a pasar la función suma a un módulo ensamblador separado.

En la séptima se volvía a incorporar el código ensamblador de suma como ensamblador en-linea.

La octava versión reducía el código ensamblador en-linea al cuerpo del bucle \textbf{for}.

Y por último, en la novena se creaban \textbf{tres alternativas a suma} cronometrando sus tiempos de ejecución.

\subsection {Segunda sesión}
Realización de las preguntas de autocomprobación desde suma\_01 hasta suma\_09.

Estudio de las distintas versiones de \textbf{popcount} a realizar así
como una pequeña estructuración y organización de cada uno de ellos.

Desarrollo de cada uno de los códigos correspondientes a \textbf{popcount}, por medio
del guión de prácticas con la ayuda de las transparencias de clase y del libro de la asignatura
para llegar a sus soluciones.

Toma de los resultados obtenidos en las ejecuciones y elaboración de un estudio por 
medio de una hoja de cálculo con sus correspondientes gráficas finales.

\subsection {Tercera sesión}

Estudio de las distintas versiones de \textbf{parity} a realizar así
como una pequeña estructuración y organización de cada uno de ellos.

Desarrollo de cada uno de los códigos correspondientes a \textbf{parity}, por medio
del guión de prácticas con la ayuda de las transparencias de clase y del libro de la asignatura
para llegar a sus soluciones.

Toma de los resultados obtenidos en las ejecuciones y elaboración de un estudio por 
medio de una hoja de cálculo con sus correspondientes gráficas finales.

\newpage
%----------------------------------------------------------------------------------------
%	Ejercicio 4.0
%----------------------------------------------------------------------------------------

\section{Ejercicio 4.0: Preguntas de autocomprobación (suma\_01-suma\_09)}
\subsection{Sesión de depuración suma\_01\_S\_cdecl}
\textbf{\textit{4.¿Qué modos de direccionamiento usa la instrucción add (\%ebx,\%edx,4), \%eax?}}
\\

La instrucción usa el modo de direccionamiento inmediato.

Los componentes son: add(rt, rs, imm),escala.

\subsection{Sesión de depuración suma\_02\_S\_libC}
\textbf{\textit{1.¿Qué error se obtiene si no se añade –lc al comando de enlazar? ¿Qué tipo de error es? (en tiempo de ensamblado, enlazado, ejecución...)
}}
\\

Se obtienen varios errores de referencias a funciones que no están definidas.

Se trata de un error en tiempo de enlazado.

\subsection{Sesión de depuración suma\_03\_SC}
\textbf{\textit{2.¿Qué diferencia hay entre los comandos Next y Step?}}
\\

La diferencia está en que, por ejemplo, si se encuentra en una llamada a función y se ejecuta el comando \textbf{Next}, la función se ejecutará y volverá. En cambio con \textbf{Step} se irá a la primera línea de esa función.


\subsection{Sesión de depuración suma\_04\_SC}
\textbf{\textit{3. Otras diferencias están en el manejo de pila. Explicar dichas diferencias.}}
\\

La función \textbf{\_\_printf\_chk flag} comprobará el desbordamiento de pila antes de calcular un resultado, dependiendo del valor del parámetro \textit{flag}. Si se prevé un desbordamiento, la función se cancelará y el programa se saldrá del programa que la llama.

\subsection{Sesión de depuración suma\_05\_SC}
\textbf{\textit{5. Se aplica una máscara 
		al punte ero de pila. ¿Cuál, 
		y  qué efecto produce? (Pista: alineamiento). 
		Nosotros usam
		mos .int para de clarar enteros y arrays. ¿Qué usa 
		gcc? Nosotros usamos el contador de posiciones y aritmé
		ética de etiquetas para calcular la longitud del array. ¿Qué usa gcc? Nosotros hemos usado push para introducir argumentos en pila, ¿Cuál usa gcc? 
}}
\\

Máscara de pila: \textbf{leaq	0(,\%rax,4), \%rdx}

Para declarar enteros y arrays gcc usa \textbf{.long}

gcc para introducir argumentos en pila utiliza \textbf{pushq}

\subsection{Sesión de depuración suma\_07\_Casm}
\textbf{\textit{Por motivos estéticos, a veces se terminan las líneas ASM con $ "\textbackslash n" $ y otras con $ "\textbackslash n \textbackslash t" $.  
		¿Por qué en este caso apenas se ha usado $ "\textbackslash t" $?}}
	\\
	
	No se ha usado apenas $ "\textbackslash t" $ para que cuando se imprima el código para su depuración o para su lectura resulte más cómodo.

\subsection{Sesión de depuración suma\_08\_Casm}
\textbf{\textit{Si res es variable de salida, ¿por qué se le ha indicado restricción $ "+r" $, en lugar de $ "=r" $? 
}}
\\

Se le ha indicado la restricción $ "+r" $ porque ésta le dice a gcc qué necesita para cargar un registro con el valor, es decir, significa que este operando es leído y escrito por la instrucción.

Cuando el compilador arregla los operandos para satisfacer las restricciones, necesita saber qué operandos son leídos por la instrucción y que son cuáles por ella. \textbf{'='} Identifica un operando que solamente se encuentra escrito, sin embargo \textbf{'+'} identifica un operando que es escrito y a la vez leído.

\newpage

\section{Ejercicio 4.1: Calcular la suma de bits de una lista de enteros sin signo}
\subsection{Primera versión}
\begin{lstlisting}
// Primera versión, previsiblemente con peores prestaciones.
int popcount1(unsigned* array, int len)
{
	int  i,j,  res=0;
	for (i=0; i<len; i++){
		unsigned x =array[i];
		for (j=0; j<WSIZE;j++){     // Recorre los bits
			res += x & 0x1;         	// Aplica la máscara 0x1
			x >>=1;                 	// Desplaza a la derecha
		}
	}
	return res;
}
\end{lstlisting}

\subsection{Segunda versión}
\begin{lstlisting}
int popcount2(unsigned* array, int len)
{
	int  i, res=0;
	unsigned x;
	for (i=0; i<len; i++){
		x=array[i];
		do{
			res += x & 0x1;         // Aplica la máscara 0x1
			x >>=1;                 // Desplaza a la derecha
		}while(x);                // Recorre los bits
	}
	return res;
}
\end{lstlisting}

\newpage

\subsection{Tercera versión}
\begin{lstlisting}
// Traduce el bucle interno while por código ASM.
int popcount3(unsigned* array, int len){
	int res=0,i;
	unsigned x;
	
	for(i=0; i<len; i++){
		x=array[i];
		asm("\n"
			"ini3:		\n\t"        // seguir mientras que x!=0
			"shr %[x]	\n\t"        // LSB en CF
			"adc $0, %[r] \n\t"    // suma con acarreo
			"cmp $0,%[x] \n\t"     // compara
			"jne	ini3 \n\t"       // salto, modificando CF y ZF
			: [r]"+r" (res)        // e/s: añadir a lo acumulado por el momento
			: [x]"r" (x)           // entrada: valor elemento
		);
	}
	return res;
}
\end{lstlisting}

\subsection{Cuarta versión}
\begin{lstlisting}
int popcount4(unsigned* array, int len){
	int i,j;
	int result = 0;
	unsigned n;
	
	for(i = 0; i < len; i++){
		int res = 0;
		n = array[i];
		for(j = 0; j < 8; j++){     // Aplica la máscara a cada elemento.
			res += n & 0x01010101;    // Acumula los bits de cada byte.
			n >>= 1;
		}
		// Suma en árbol los 4B
		res += (res >> 16);
		res += (res >> 8);
		result += (res & 0xff);
	}
	return result;
}
\end{lstlisting}

\subsection{Quinta versión}
\begin{lstlisting}
// Versión SSSE3 (pshufb)
int popcount5(unsigned* array, int len) {
	int i;
	int res, result = 0;
	int SSE_mask[] = { 0x0f0f0f0f, 0x0f0f0f0f, 0x0f0f0f0f, 0x0f0f0f0f };
	int SSE_LUTb[] = { 0x02010100, 0x03020201, 0x03020201, 0x04030302 };
	
	if (len & 0x3)
		printf("leyendo 128b pero len no múltiplo de 4?n");
	for (i = 0; i < len; i += 4) {
		asm("movdqu        %[x], %%xmm0 \n\t"
		"movdqa  %%xmm0, %%xmm1 \n\t"       // dos copias de x
		"movdqu    %[m], %%xmm6 \n\t"       // máscara
		"psrlw           $4, %%xmm1 \n\t"
		"pand    %%xmm6, %%xmm0 \n\t"       // xmm0 - nibbles inferiores
		"pand    %%xmm6, %%xmm1 \n\t"       // xmm1 - nibbles superiores
		
		"movdqu    %[l], %%xmm2 \n\t"       // como pshufb sobrescribe LUT
		"movdqa  %%xmm2, %%xmm3 \n\t"       // queremos 2 copias
		"pshufb  %%xmm0, %%xmm2 \n\t"       // xmm2 = vector popcount inferiores
		"pshufb  %%xmm1, %%xmm3 \n\t"       // xmm3 = vector popcount superiores
		
		"paddb   %%xmm2, %%xmm3 \n\t"       // xmm3 - vector popcount bytes
		"pxor    %%xmm0, %%xmm0 \n\t"       // xmm0 = 0,0,0,0
		"psadbw  %%xmm0, %%xmm3 \n\t"       // xmm3 = [pcnt bytes0..7|pcnt bytes8..15]
		"movhlps %%xmm3, %%xmm0 \n\t"       // xmm3 = [       0      |pcnt bytes0..7 ]
		"paddd   %%xmm3, %%xmm0 \n\t"       // xmm0 = [   no usado   |pcnt bytes0..15]
		"movd    %%xmm0, %[res] \n\t"
		: [res]"=r" (res)
		: [x]  "m"  (array[i]),
		[m]  "m"  (SSE_mask[0]),
		[l]  "m"  (SSE_LUTb[0])
		);
		result += res;
	}
	return result;
}
\end{lstlisting}

\subsection{Sexta versión}
\begin{lstlisting}
// Versión SS4.2
int popcount6(unsigned* array, int len) {
	int i;
	unsigned x;
	int val, result=0;
	
	for(i=0; i<len; i++){
		x=array[i];
		asm("popcnt %[x], %[val]"
		: [val] "=r" (val)
		:   [x]  "r" (x)
		);
		result += val;
	}
	return result;
}
\end{lstlisting}

\newpage

\subsection{Mediciones:	cronometrar las	distintas versiones con	-O0, -O1 y -O2}



\begin{figure}[H] %con el [H] le obligamos a situar aquí la figura
	\centering
	\includegraphics[scale=0.6]{capturas/ej0_pop.png} 
	\caption{Ejecución del programa \textbf{popcount} con \textbf{-O0}} 
	\label{fig:figura2}
\end{figure}

\begin{figure}[H] %con el [H] le obligamos a situar aquí la figura
	\centering
	\includegraphics[scale=0.6]{capturas/ej1_pop.png} 
	\caption{Ejecución del programa \textbf{popcount} con \textbf{-O1}} 
	\label{fig:figura3}
\end{figure}

\begin{figure}[H] %con el [H] le obligamos a situar aquí la figura
	\centering
	\includegraphics[scale=0.6]{capturas/ej2_pop.png} 
	\caption{Ejecución del programa \textbf{popcount} con \textbf{-O2}} 
	\label{fig:figura4}
\end{figure}

\newpage

\subsection{Representación de los resultados}
\begin{figure}[H] %con el [H] le obligamos a situar aquí la figura
	\centering
	\includegraphics[scale=0.4]{capturas/tablas_pop.png} 
	\caption{Resultado de las ejecuciones de \textbf{popcount}} 
	\label{fig:figura5}
\end{figure}

\begin{figure}[H] %con el [H] le obligamos a situar aquí la figura
	\centering
	\includegraphics[scale=0.6]{capturas/medias_pop.png} 
	\caption{Resumen de los promedios} 
	\label{fig:figura6}
\end{figure}

\begin{figure}[htbp]
	\centering
	\subfigure[Completa]{\includegraphics[scale=0.35]{capturas/graf_pop1.png}}
	\subfigure[Reducida para apreciar el factor mejora]{\includegraphics[scale=0.35]{capturas/graf_pop2.png}}
	\caption{Gráficas de las ejecuciones de \textbf{popcount}} \label{fig:lego}
\end{figure}


\newpage

\section{Ejercicio 4.2: Calcular la suma de paridades de una lista de entero sin signo}
\subsection{Primera versión}
\begin{lstlisting}
// Primera versión
int parity1(unsigned* array, int len)
{
	int  i,j,  res=0,val;
	for (i=0; i<len; i++){          // Recorre el array
		val=0;
		unsigned x =array[i];
		for (j=0; j<WSIZE;j++){     // Recorre los bits
			val ^= x & 0x1;         // Aplica la máscara y acumula lateralmente los bits
			x >>=1;                 // Desplaza a la derecha para extraer y acumular bits
		}
		res+=val;
	}
	return res;
}
\end{lstlisting}

\subsection{Segunda versión}
\begin{lstlisting}
// Segunda versión
int parity2(unsigned* array, int len)
{
	int  i,res=0,val;
	unsigned x;
	for (i=0; i<len; i++){
		val=0;
		x=array[i];
		do{
			val ^= x & 0x1;     // Aplica la máscara y acumula lateralmente los bits
			x >>=1;             // Desplaza a la derecha para extraer y acumular bits
		}while(x);              // Recorre el array
		res+=val;
	}
	return res;
}
\end{lstlisting}

\subsection{Tercera versión}
\begin{lstlisting}
// Tercera versión: Adapta la segunda versión para el array completo
int parity3(unsigned* array, int len) {
	int i;
	unsigned x;
	int result = 0;
	for (i = 0; i < len; i++) {
		x = array[i];
		int res=0;
		while (x) {
			res ^= x;            // Acumula lateralmente los bits
			x >>= 1;             // Desplaza a la derecha para extraer y acumular bits
		}
		result += res & 0x1;     // Aplicar la máscara al acumular con result.
	}
	return result;
}
\end{lstlisting}

\subsection{Cuarta versión}
\begin{lstlisting}
// Cuarta versión: Traduce el bucle intero por ASM
	int parity4(unsigned* array, int len) {
	int i,res;
	unsigned x;
	int result = 0;
	for (i = 0; i < len; i++) {
		x = array[i];
		res = 0;
		asm(
			"ini3:				\n\t"   // seguir mientras que x!=0
			"xor %[x], %[v]		\n\t"
			"shr $1, %[x]		\n\t"   // LSB en ZF
			"test %[x], %[x]	\n\t"
			"jnz ini3			\n\t"   // salto, modificando CF y ZF
			: [v]"+r"(res)              // e/s: entrada 0, salida paridad elemento
			
			: [x]"r"(x)                 // entrada: valor elemento
		);
		result += res & 0x1;            // Aplicar la máscara al acumular con result.
	}
	return result;
}
\end{lstlisting}

\subsection{Quinta versión}
\begin{lstlisting}
// Quinta versión: Suma en árbol
int parity5(unsigned* array, int len) {
	int i, j;
	int result = 0;
	unsigned x;
	for (i = 0; i < len; i++) {
		x = array[i];
		// Somete a cada elemento a XOR y desplazamientos
		// sucesivos a mitdad de la distancia cada vez
		for (j = 16; j > 0; j /= 2)
			x ^= x >> j;
		result += x & 0x01;     // Aplica la máscara al valor final.
	}
	return result;
}
\end{lstlisting}

\newpage

\subsection{Sexta versión}
\begin{lstlisting}
// Sexta versión: Traduce el bucle for por ASM
int parity6(unsigned* array, int len) {
	int j,result = 0;;
	unsigned x = 0;
	
	for (j = 0; j < len; j++) {
		x = array[j];
		asm(
			"mov	%[x], %%edx	     \n\t" // Sacar copia para XOR
			"shr	$16,	%%edx	 \n\t"
			"xor	%[x],	%%edx	 \n\t" // PF, señalando la paridad par de los 8 bits inferiores
			"xor	%%dh,	%%dl	 \n\t"
			"setpo  %%dl	         \n\t"
			"movzx	%%dl,	%[x]	 \n\t" // Devuelve en 32 bits
			: [x] "+r" (x)                 // e/s entrada valor elemento, salida paridad
			:
			: "edx"                        // clobber
		);
		result += x;
	}
	return result;
}
\end{lstlisting}

\newpage

\subsection{Mediciones:	cronometrar las	distintas versiones con	-O0, -O1 y -O2}



\begin{figure}[H] %con el [H] le obligamos a situar aquí la figura
	\centering
	\includegraphics[scale=0.6]{capturas/ej0_par.png} 
	\caption{Ejecución del programa \textbf{parity} con \textbf{-O0}} 
	\label{fig:par1}
\end{figure}

\begin{figure}[H] %con el [H] le obligamos a situar aquí la figura
	\centering
	\includegraphics[scale=0.6]{capturas/ej1_par.png} 
	\caption{Ejecución del programa \textbf{parity} con \textbf{-O1}} 
	\label{fig:par3}
\end{figure}

\begin{figure}[H] %con el [H] le obligamos a situar aquí la figura
	\centering
	\includegraphics[scale=0.6]{capturas/ej2_par.png} 
	\caption{Ejecución del programa \textbf{parity} con \textbf{-O2}} 
	\label{fig:par4}
\end{figure}

\newpage

\subsection{Representación de los resultados}
\begin{figure}[H] %con el [H] le obligamos a situar aquí la figura
	\centering
	\includegraphics[scale=0.4]{capturas/tablas_par.png} 
	\caption{Resultado de las ejecuciones de \textbf{parity}} 
	\label{fig:par5}
\end{figure}

\begin{figure}[H] %con el [H] le obligamos a situar aquí la figura
	\centering
	\includegraphics[scale=0.6]{capturas/medias_par.png} 
	\caption{Resumen de los promedios} 
	\label{fig:par6}
\end{figure}

\begin{figure}[htbp]
	\centering
	\subfigure[Completa]{\includegraphics[scale=0.35]{capturas/graf_par2.png}}
	\subfigure[Reducida para apreciar el factor mejora]{\includegraphics[scale=0.35]{capturas/graf_par1.png}}
	\caption{Gráficas de las ejecuciones de \textbf{parity}} \label{fig:par7}
\end{figure}

\newpage

\end{document}
