\input{preambuloSimple.tex}



%----------------------------------------------------------------------------------------
% DOCUMENTO
%----------------------------------------------------------------------------------------

\begin{document}
	
{\LARGE Desactivación Bomba de María Camarero \par}
\vspace{5mm}
{\large Estructura de Computadores - Grupo C3 \par}
\vspace{3mm}
{\large \textit{Mario Rodríguez Ruiz} \par}	
	
	
%----------------------------------------------------------------------------------------
%	Cuestión 1
%----------------------------------------------------------------------------------------

\section{Contraseña}

Para averiguar la contraseña se ha utilizado el depurador \textbf{DDD} realizando los siguientes pasos:
\\

En primer lugar se ha puesto un punto de ruptura en la llamada a \textbf{fgets}, que es cuando se le pide al usuario por pantalla que ingrese una contraseña.\\

Una vez funcionando el programa y detenido ahí se ha metido una contraseña al azar para avanzar.

\begin{figure}[H]
	\centering
	\includegraphics[scale=0.9]{capturas/figura1.png} 
	\caption{Examine Memory desde DDD} 
	\label{fig:figura1}
\end{figure}

Justo después del guardado de la contraseña introducida, se hace un \textbf{push} "sospechoso".
Ha sido el momento de examinar los datos en memoria mediante la opción \textbf{Data} $ \rightarrow $ \textbf{Memory}:

Poniendo Examine a 1, tipo \textbf{string} en \textbf{bytes} desde la dirección \textbf{0x804a03c} y haciendole un \textbf{print} ha mostrado la contraseña tal y como se ve en la Figura \ref{fig:figura1}.
\\

Contraseña: \textbf{ec2016}
\newpage

%----------------------------------------------------------------------------------------
%	
%----------------------------------------------------------------------------------------

\section{Código}
Para averiguar la contraseña se ha utilizado el depurador \textbf{DDD} realizando los siguientes pasos:
\\

Llegando a la sección del programa que se encarga de procesar y validar el código, se ha puesto un punto de ruptura en esta parte para futuros intentos.\\

Como se ve en la Figura \ref{fig:figura2} el programa hace una llamada a una función con nombre \textbf{desactivar}, por lo que mediante \textbf{stepi} la ejecución seguirá su curso adentrándose en ésta.\\
\begin{figure}[H]
	\centering
	\includegraphics[scale=1]{capturas/figura2.png} 
	\caption{Examine Memory desde DDD para el código} 
	\label{fig:figura2}
\end{figure}
Una vez dentro de dicha función, tal y como se muestra en la Figura \ref{fig:figura3}, se pone un punto de ruptura para futuros intentos y a continuación se examina.


\begin{figure}[H]
	\centering
	\includegraphics[scale=1]{capturas/figura3.png} 
	\caption{Examine Memory desde DDD para el código} 
	\label{fig:figura3}
\end{figure}

Mediante la ayuda del estado de los registros (\textbf{Status}$ \rightarrow $\textbf{Registers}) y avanzando una a una cada instrucción se puede ver cómo van modificando sus valores.
\newpage



El código de prueba introducido ha sido \textbf{1111}, sin embargo $ " $extrañamente$ " $ se ha visto cómo se ha modificado su valor convirtiéndose ahora en \textbf{-905}, como puede verse en la Figura \ref{fig:figura4}.

Es ahora cuando se hace una comparación de este nuevo valor con el valor que contiene \textbf{\%eax} que es cero.
\begin{figure}[H]
	\centering
	\includegraphics[scale=0.8]{capturas/figura4.png} 
	\caption{Examine Memory desde DDD para el código} 
	\label{fig:figura4}
\end{figure}
Aquí es cuando se aprecia que se ha realizado una resta sobre el valor introducido y que su resultado tiene que ser cero. Por lo que sumando el valor introducido como prueba (1111) y el resultado después de la modificación (905) aparece la solución al problema: 1111+905=2016
\\

Por tanto, el código es \textbf{2016}.

%----------------------------------------------------------------------------------------
%	Referencias
%----------------------------------------------------------------------------------------
%------------------------------------------------

\bibliography{citas} %archivo citas.bib que contiene las entradas 
\bibliographystyle{plain} % hay varias formas de citar

\end{document}
